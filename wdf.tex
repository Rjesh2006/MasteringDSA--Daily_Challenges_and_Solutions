\documentclass[a4paper,10pt]{article}
\usepackage{graphicx}
\usepackage[margin=0.75in]{geometry} 
\input{RAJH}


\begin{document}

\title{Ramanujan: The Mathematician}
\author{\\ Delhi University}
\date{}
\maketitle



\tableofcontents
\vspace{10pt}

\section*{Abstract}
This paper explores the contributions of Srinivasa Ramanujan, a mathematician who made substantial contributions to mathematical analysis, number theory, infinite series, and continued fractions. His groundbreaking work has influenced many areas of mathematics.


\section{Introduction}
Srinivasa Ramanujan was an Indian mathematician whose life and work have left a significant mark on the world of mathematics. Despite lacking formal education, he contributed numerous theorems and formulas that continue to influence mathematical theory.

\section{Contributions}
Ramanujan's work spanned a variety of topics in mathematics, from number theory to infinite series.

\subsection{Number Theory}
One of his most famous contributions is the Hardy-Ramanujan number, which highlights his deep understanding of number theory.

\subsection{Infinite Series}
Ramanujan developed many new and innovative series, including those related to the value of Pi.

\section{Legacy}
Ramanujan's influence is still felt today, and his formulas continue to be used in modern mathematical research.

\section*{Mathematician Image}
\begin{center}
\includegraphics[width=0.3\textwidth]{} 
\begin{figure}
    \centering
    \includegraphics[width=0.5\linewidth]{image.png}
    \caption{Enter Caption}
    \label{fig:enter-label}
\end{figure}

\end{center}

\section{Conclusion}
Srinivasa Ramanujan's contributions to mathematics are immeasurable. His work has inspired countless mathematicians, and his legacy endures.

\section*{References}
1. G.H. Hardy, *A Mathematician's Apology*, Cambridge University Press, 1940. \\
2. Bruce C. Berndt, *Ramanujan's Notebooks: Part I*, Springer, 1985. \\
3. Robert Kanigel, *The Man Who Knew Infinity: A Life of the Genius Ramanujan*, Scribner, 1991.

\end{document}
