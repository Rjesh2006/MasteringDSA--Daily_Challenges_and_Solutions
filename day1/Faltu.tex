\documentclass{scrartcl}
\usepackage{silence}
\usepackage{fontspec}
\WarningFilter{latex}{Command \InputFileExists}
\usepackage{xunicode}
\usepackage{fontawesome}
\usepackage{devanagari}
\usepackage{polyglossia}
\usepackage[hyphens]{url}
\setmainlanguage{english}
\setotherlanguages{arabic,hindi,sanskrit,greek,thai}

\setmainfont{Noto Serif}
\setsansfont{Noto Sans}
\setmonofont{Noto Mono}

\newfontfamily\arabicfont[Script=Arabic]{Noto Naskh Arabic}
\newfontfamily\devanagarifont[Script=Devanagari]{Noto Serif Devanagari}
\newfontfamily\thaifont[Script=Thai]{Noto Serif Thai}

\usepackage[space]{xeCJK}
\setCJKmainfont{Noto Serif CJK SC}
\newCJKfontfamily\japanesefont{Noto Serif CJK JP}
\newCJKfontfamily\koreanfont{Noto Serif CJK KR}

\begin{document}

\title{Multilingual Document}
\author{Your Name}
\date{\today}
\maketitle

\section{Introduction (English)}
This document demonstrates the use of multiple languages in LaTeX using the \texttt{polyglossia} package.

\section{Arabic}
\begin{arabic}
مرحبا بكم في هذا المستند متعدد اللغات.
\end{arabic}

\section{Hindi}
\begin{devanagari}
यह एक बहुभाषी दस्तावेज़ का उदाहरण है।
\end{devanagari}

\section{Sanskrit}
\begin{sanskrit}
अयं एकः बहुभाषीय दस्तावेजस्य उदाहरणम् अस्ति।
\end{sanskrit}

\section{Greek}
\textgreek{Αυτό είναι ένα παράδειγμα πολυγλωσσικού εγγράφου.}

\section{Thai}
\begin{thaifont}
นี่คือตัวอย่างของเอกสารหลายภาษา
\end{thaifont}

\section{Chinese}
你好,这是一个多语言文档的示例。

\section{Japanese}
\japanesefont{これは多言語ドキュメントの例です。}

\section{Korean}
\koreanfont{이것은 다국어 문서의 예입니다.}

\end{document}
